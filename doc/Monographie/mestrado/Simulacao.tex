\chapter{Simulação Distribuída de Eventos Discretos}

Parágrafo 1: Revisar a déia inicial sobre o que é simulação, e dizer que a simulação é baseada em um modelo do problema

Parágrafo 2: Dizer que a simulação é um processo caro, e que é conveniente dividir entre vários computadores (distribuir a simulação)

Parágrafo 3: Apresentar problemas como sincronização, comunicação, balanceamento de cargas, etc. devido à distribuição.

Parágrafo 4: Dividir o sistema em Processo lógicos e eventos discretos. Rever o conceito de timestamp de cada evento.

Parágrafo 5: Dizer que cada processo lógico possui um relógio interno

Parágrafo 6: Definir comunicação através de troca de mensagens

Parágrafo 7: Definir o conceito de sincronização (timestamp das mensagens e LVT)


\section{Categorias de protocolos de simulação}

Parágrafo 1: breve discussão sobre as categorias de protocolos, explicando as principais diferenças

\subsection{Protocolos conservaticos}

Parágrafo 1: citar Srinivasan and Reynolds, explicando o conceito básico de um protocolo conservativo

Parágrafo 2: citar as implementacoes de [chandy e misra] e bryant, citando a necesidade de canais estaticos

Parágrafo 3: citar problema de bloqueio por esperar mensagem que contém um lvt inferior, citar também como contornar esse deadlock

Parágrafo 4: explicar porque é difícil resolver o problema de deadlock

Parágrafo 5: citar a desvantagem do conservativo por não aproveitar todo o paralelismo

Parágrafo 6: conclusão pessoal de porque não utilizar o protocolos conservativos

\subsection{Protocolos Otimistas}

Parágrafo 1: explicação inicial do conceito de protocolo otimista e citar o conceito de rollback

Parágrafo 2: Citar o time warp, contextualizando como foi desenvolvido, em qual época e por quem

Parágrafo 4: Citar o Rollback Solidário como alternativa ao Time Warp

\section{O protocolo \textit{Time Warp}}

Parágrafo 1: Explica por que o Time Warp é otimista, citando o fato de nao impedir a ocorrência de erros de causa e efeito

Parágrafo 2: Citar como um erro de causa e efeito pode acontecer, e o que deve ser feito quando isso acontece: o rollback

Parágrafo 4: Citar implementações do protocolo

Parágrafo 5: Explicar o comportamento individual de um processo lógico (tirar o eventos com menor timestamp da lista de eventos futuros, etc.)

\subsection{Detecção e Tratamento de Inconsistências}

Parágrafo 1: Mostrar que o sistema pode receber mensagens que causam erro de causa e efeito.

Parágrafo 2: Explicar o comportamento ao receber uma mensagem straggler.

Parágrafo 3: definir a diferença entre rollback primário e rollback secundário

\subsection{Anti-Mensagens}

Parágrafo 1: Explicar o que é uma antimensagem

Parágrafo 2: mostrar que antimensagens podem se referir a um evento já processado, ou a um evento que ainda esta na fila de eventos futuros

Parágrafo 3: Tratar o caso da antimensagem chegar antes da mensagem

\subsection{Considerações Finais}


\section{O protocolo \textit{Rollback} Solidário}

Parágrafo 1: Apresentar a principal diferença do rollback solidário

Parágrafo 2: explicar que, em caso de erro de causa e efeito, todos os processos executam rollback em conjunto

Parágrafo 3: introduzir o conceito de checkpoint, checkpoint global e checkpoint local.

\subsection{Comportamento Geral do Protocolo Rollback solidário}

Trazer uma idéia básica do Rollback solidário em linhas gerais

\subsection{Estados Locais e Estados Globais}

Parágrafo 1: Definição de um estado local como sendo os valores das variáveis do processo em questão em um determinado tempo

Parágrafo 2: Definição de estado global como sendo um conjunto de estdaos de cada processo.

\subsection{Cortes Globais Consistentes}

Parágrafo 1: Apresentar o conceito de precedência causal

Parágrafo 2: Definir corte 

Parágrafo 3: definir corte consistente utilizando precedência causal

\subsection{Relógios Lógicos e Relógios Vetoriais}

Parágrafo 1: Apresentar o conceito de relógio lógico

Parágrafo 2: Mostrar que o relógio lógico não representa a precedência causal em um sistema distribuído

Parágrafo 3: Apresentar o conceito de relógio vetorial

\subsection{Checkpoints Globais Consistentes}

Parágrafo 1: Definir o que é um checkpoint: um ponto de retorno que foi, de alguma forma, armazenado de forma persistente

Parágrafo 2: a necessidade de se garantir que um checkpoint seja consistente

Parágrafo 3: associação de checkpoint consistente com corte consistente

\subsection{Obtenção de Checkpoint Semi-Síncrono}

Parágrafo: introduzir o processo observador

Parágrafo: Introduzir o vetor de dependência

Parágrafo: Mostrar como a troca de mensagem deve levar o vetor de dependências, e como este deve ser atualizado no processo que recebeu a mensagem

Parágrafo: mostrar como o processo observador monta uma linha de recuperação através da matriz de dependências

Parágrafo: demonstrar que um checkpoint global consistente é formado por checkpoints locais que não possuem relação causal entre si


\subsection{Tratamento dos Rollbacks na Abordagem Semi-Síncrona}

Parágrafo: Explicar o comportamento de um processo ao receber uma mensagem straggler 

Parágrafo: Falar sobre a atuação do processo observador (escolher a linha de retorno, avisar o rollback à todos em broadcast, etc.)

Parágrafo: Falar sobre o reinício da simulação após o Rollback

\subsection{Considerações Finais}


\section{Balanceamento de cargas}

Parágrafo: apresentar o balanceamento de carga como fator determinante no desempenho de uma simulação distribuída

Parágrafo: Introduzir o conceito de escalonamento de processos como meio de prover o balanceamento de cargas

Parágrafo: por fim mostrar que para prover o balanceamento de cargas temos que possibilitar que um processo lógico migre de um nó do sistema para outro

\subsection{O Uso de Agentes Móveis Para Prover Mobilidade}

Parágrafo: O que é um agente móvel

Parágrafo: Falar sobre a implementação feita por Antonio, Walbon e Takahashi - 2010

Parágrafo: Outras implementações feitas usando agentes móveis

\subsection{Requisitos Para Mobilidade de um Processo Lógico}

Parágrafo: explicar quais as funcionalidades de agentes móveis que precisamos (migração)

Parágrafo: Abstrair o conceito de ambiente e processo lógico: um ambiente abriga diversos processos lógicos

Parágrafo: Mostrar a idéia de migrar um processo de um ambiente para outro, a fim de prover o balanceamento

\subsection{Considerações Finais}
