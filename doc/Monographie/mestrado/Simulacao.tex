\chapter{Simulação Distribuída de Eventos Discretos}

% Parágrafo 1: Revisar a déia inicial sobre o que é simulação, e dizer que a simulação é baseada em um modelo do problema

A simulação é uma técnica que permite prever e visualizar o comportamento de sistemas reais a partir de modelos matemáticos. As aplicações da simulação abrangem diversos benefícios, tais como: a possibilidade de antever possíveis problemas ou comportamentos indesejáveis de um sistema, auxílio na tomada de decisão sem a necessidade de intervir no sistema real, facilidade na manipulação e alteração dos modelos, economia de recursos (físicos e financeiros) durante a tomada de decisões, dentre outros.

% Parágrafo 2: Dizer que a simulação é um processo caro, e que é conveniente dividir entre vários computadores (distribuir a simulação)

Por demandar um intenso processamento computacional, a simulação por vezes pode se tornar uma opção demasiadamente custosa. Melhores algorítmos e computadores mais modernos são maneiras de tornar a simulação mais eficiente. Em paralelo à isto existe a possibilidade de se distribuir a simulação entre vários nós de um sistema, ou entre vários núcleos de uma máquina paralela. O uso desta técnica permite que certos processos sejam executados simultaneamente, acelerando assim a simulação.

% Parágrafo 3: Apresentar problemas como sincronização, comunicação, balanceamento de cargas, etc. devido à distribuição.

Em contrapartida, ao se dividir uma simulação entre vários nós em um sistema distribuído são criadas diversos novos requisitos ao sistema. Ítens como sincronização dos processos, comunicação entre os processos, balanceamento de cargas no sistema, dentre outros, devem ser gerenciados pela aplicação de simulação a fim de se garantir um resultado consistente do processo de simulação.

\section{Princípios da Simulação Baseada em Eventos Discretos}
% Parágrafo 4: Dividir o sistema em Processo lógicos e eventos discretos. Rever o conceito de timestamp de cada evento.

No paradigma da simulação discreta, três estruturas representam a simulação orientada a eventos:

\begin{itemize}
\item As variáveis internas do sistema;
\item Uma lista de eventos, denominada lista de eventos futuros. Esta lista abriga os eventos a serem executados;
\item Um relógio lógico global, que controla o processo de execução da simulação
\end{itemize}

O relógio lógico global armazena o chamado Tempo Virtual Global (\textit{Global Virtual Time} - GVT). O GVT representa o atual momento da simulação. O seu valor sofre incrementos discretos, e cada unidade do GVT representa uma fração real de tempo, discretizada para o modelo a ser simulado.

Cada evento a ser executado possui uma marca  de tempo (\textit{timestamp}) que determina quando, no tempo virtual do sistema, este evento deve ser executado. O programa de simulação repetidamente retira o evento com o menor \textit{timestamp} da fila de eventos futuros para processa-lo.

Para a implementação da simulação distribuída, a estrutura da simulação tradicional, inerentemente sequencial, teve que sofrer adaptações para incorporar os conceitos de computação distribuída \cite{REED-MALONY}. Para isto, o sistema passou a ser dividido em processos lógicos que representam um processo do sistema real. Uma simulação é um conjunto com $n$ processos lógicos $p_{1}, p_{2}, p_{3}, ..., p_{n}$.

% Parágrafo 5: Dizer que cada processo lógico possui um relógio interno

Assim como o sistema de simulação centralizado possui um relógio que representa o tempo global da simulação, ao se dividir a simulação em um sistema distribuído cada processo lógico possui um relógio lógico interno que representa o avanço do tempo da simulação para aquele processo. O relógio de cada processo lógico armazena o que é chamado de Tempo Virtual Local (\textit{Local Virtual Time} - LVT), e é este LVT que indica o tempo virtual de cada processo lógico.

Isso garante que os eventos a serem executados por um determinado processo lógico sejam executados no tempo determinado pelo seu \textit{timestamp}. Portanto um processo lógico, se olhado isoladamente, possui as mesmas variáveis que um sistema de simulação sequencial: uma fila de eventos a serem executados, um relógio lógico interno e as suas variáveis internas.

% Parágrafo 6: Definir comunicação através de troca de mensagens

Porém em certos momentos um processo pode necessitar comunicar-se com outro processo lógico. Por exemplo, um processo pode enviar um evento para ser executado por outro processo lógico. Esta comunicação é feita através de troca de mensagens. Um processo lógico deve ser capaz de enviar mensagens à outros processos lógicos que, por sua vez, devem ser capaz de receber estas mensagens. 

% Parágrafo 7: Definir o conceito de sincronização (timestamp das mensagens e LVT)

Assim como todo evento discreto, um evento enviado de um processo lógico à outro possui um \textit{timestamp} que indica quando este deve ser executado pelo processo lógico em questão. Porém, como os processos lógicos em um sistema distribuído não compartilham um mesmo relógio lógico (cada processo lógico possui seu próprio relógio interno), o sistema de simulação distribuída deve estar preparado para tratar situação de inconsistência da simulação.

Como cada processo lógico possui um relógio lógico interno independente, pode acontecer de um processo estar em um tempo virtual mais atrasado que os demais e enviar um evento para ser processado em um tempo que já foi superado pelo relógio lógico do processo que recebe a mensagem. Assim como ilustrado na figura~\ref{fig:strag}, o processo em questão possui seu relógio lógico no tempo $t=24$ e recebe uma mensagem para ser processada no tempo $t=11$. Como o evento a ser processado possui um \textit{timestamp} inferior ao valor do relógio interno do processo lógico, o sistema encontra-se em um momento de inconsistência. Um sistema de simulação distribuída deve ser capaz de contornar (ou evitar) tal situação.

\begin{figure}
  \centerline{\includegraphics[scale=0.4]{straggler.png}}
  \caption{Mensagem \textit{straggler}.}
\label{fig:strag}
\end{figure}

Uma mensagem que contém um evento para ser processado em um tempo pertencente ao passado lógico do processo que a recebeu é denominada Mensagem \textit{Straggler}.

\section{Categorias de Protocolos de Simulação}

% Parágrafo 1: breve discussão sobre as categorias de protocolos, explicando as principais diferenças

Para se garantir a sincronização dos processos lógicos na simulação distribuída são usados protocolos de sincronização, que garantem a consistência do sistema ao final da simulação. Os protocolos podem ser divididos em duas categorias quanto ao seu comportamento perante um inconsistência de causa e efeito no sistema: protocolos conservadores e protocolos otimistas. 

Um protocolo conservador evita que um erro de causa e efeito aconteça, garantindo que toda mensagem recebida por um processo lógico esteja dentro do tempo de execução daquele processo. Já um protocolo do tipo otimista não previne a ocorrência de inconsistências no sistema, mas provê mecanismos capazes de recuperar o sistema quando elas ocorrem.

\subsection{Protocolos Conservativos}

% Parágrafo 1: citar Srinivasan and Reynolds, explicando o conceito básico de um protocolo conservativo

Os protocolos conservativos evitam a possibilidade da ocorrência de erros de causa e efeito, ou seja, sua preocupação está em determinar quando é seguro processar um evento. Nas simulações distribuídas sincronizadas com protocolos conservativos, um processo lógico só tratará um evento se puder garantir que não chegará um outro com evento com um \textit{timestamp} inferior ao do evento a ser tratado \cite{SRINIVASAN-REYNOLDS-1999}.

% Parágrafo 2: citar as implementacoes de [chandy e misra] e bryant, citando a necesidade de canais estaticos

Os primeiro protocolos de simulação distribuídas foram baseados em abordagens conservativas. Esses protocolos, desenvolvidos independentemente pro Chandy e Misra (1979) e Bryant(1977) (também tratados neste texto por CMB), exigiam que a especificação dos canais de comunicação entre os processos fosse feita estaticamente e também que as mensagens chegassem em cada canal obedecendo uma ordem crescente dos seus \textit{timestamps}.

Como os canais de comunicação são previamente conhecidos, uma vez que foram previamente especificados antes do início da simulação, é possíıvel, para cada processo, determinar o tempo virtual dos seus processos vizinhos, já que cada mensagem recebida está rotulada com o relógio local do processo emissor. Com essa informação, os processos podem tratar os eventos que possuam tempo de ocorrência menor que o tempo virtual dos canais de comunicação que chegam ao processo.

% Parágrafo 3: citar problema de bloqueio por esperar mensagem que contém um lvt inferior, citar também como contornar esse deadlock

% Parágrafo 4: explicar porque é difícil resolver o problema de deadlock

Um problema pode surgir quando não há mensagem trafegando através de um determinado canal de comunicação. Quando isto ocorre, não há como um processo obtero o \textit{LVT} daquele canal, o que pode levar à um \textit{deadlock}. Para contornar tal efeito é necessário lançar mão de mecanismos, como as mensagens nulas proposta por \cite{CMB1}, que força a atualização do LVT dos canais que estão vazios.

% Parágrafo 5: citar a desvantagem do conservativo por não aproveitar todo o paralelismo

Segundo Fujimoto~\cite{FUJIMOTO} a maior desvantagem dos mecanismos conservativos consiste no fato de que estes não podem explorar totalmente o paralelismo disponível na aplicação, uma vez que frequentemente os processos lógicos precisam interromper sua execução a fim de aguardar a chegada de um novo evento, ou mesmo uma mensagem nula.

\subsection{Protocolos Otimistas}

% Parágrafo 1: explicação inicial do conceito de protocolo otimista e citar o conceito de rollback

Diferentemente dos protocolos conservativos, os protocolos otimistas não possuem nenhum mecanismo que evite a ocorrência de uma anomalia de causa e efeito. O que acontece em um protocolo otimista é que, uma vez detectado o recebimento de uma mensagem \textit{straggler}, o protocolo deve ser capaz de desfazer a simulação até um tempo virtual onde a mensagem recebida não representa uma inconsistência no sistema, e continuar a simulação a partir deste ponto.

% Parágrafo 2: Citar o time warp, contextualizando como foi desenvolvido, em qual época e por quem

Um dos mais conhecidos protocolos otimistas é o Time Warp, que possui algumas implementações, tais como \textit{Jade Time Warp} \cite{BAEZNER1}, o sistema \textit{SPEE-DES} (\textit{Synchronous Parallel Environment for Emulation and Discrete Event Simulation})\cite{STEINMAN92}, o \textit{WARPED} \cite{WARPED} e o \textit{Georgia Tech Time Warp (GTW)} \cite{DAS94}. O \textit{Time Warp} foi originalmente proposto por Jefferson~\cite{JEFFERSON}.

% Parágrafo 4: Citar o Rollback Solidário como alternativa ao Time Warp
Desenvolvido por~\cite{EDMARMO}, o protocolo \textit{Rollback} Solidário também utiliza a abordagem otimista para sincronizar os processos lógicos em uma simulação distribuída.

\section{O Protocolo \textit{Time Warp}}

% Parágrafo 3: Explicar o comportamento individual de um processo lógico (tirar o eventos com menor timestamp da lista de eventos futuros, etc.)
Assim como na simulação sequencial, um processo lógico na simulação distribuída efetua repetidamente a ação de retirar um evento da fila de eventos futuros e tratá-lo. 

O tratamento de um evento pode ou não (dependendo do modelo simulado) gerar novos eventos para serem colocados na fila de eventos futuros. Uma vez gerado um novo evento, caso  ele seja destinado ao mesmo processo lógico, este é simplesmente inserido na fila de eventos futuros, obedecendo o seu \textit{timestamp}.

Porém, caso a execução de um evento resulte num novo evento a ser tratado por um processo lógico distinto, o processo cuja execução originou o novo evento deve encaminhar este novo evento para o seu processo correspondente. Para isto os processos lógicos se comunicam através de troca de mensagens. Quando um processo $p_{2}$ necessita encaminhar um evento para ser tratado pelo processo $p_{1}$, isto é feito através do envio de uma mensagem de $p_{2}$ para $p_{1}$ (figura~\ref{fig:envia_mensagem}).

\begin{figure}
  \centerline{\includegraphics[scale=0.4]{envia_mensagem.png}}
  \caption{Processo $p_{2}$ envia uma mensagem com \textit{timestamp} $t=5$ para o processo $p_{1}$.}
\label{fig:envia_mensagem}
\end{figure}

% Parágrafo 1: Explica por que o Time Warp é otimista, citando o fato de nao impedir a ocorrência de erros de causa e efeito
Como os processos lógicos executam de forma independente entre si, não existe nenhuma garantia que os processos $p_{1}$ e $p_{2}$ possuam o mesmo tempo virtual no momento da troca de mensagem. Caso o LVT do processo $p_{1}$ seja superior ao valor do LVT do processo $p_{2}$, existe uma garantia de que o \textit{timestamp} da mensagem enviada por $p_{1}$ também seja superior ao valor do LVT de $p_{2}$, e isso garante a consistência da simulação neste caso. Mas o mesmo não ocorre quando o LVT de $p_{1}$ é inferior ao LVT de $p_{2}$. Neste caso existe a possibilidade de que o \textit{timestamp} da mensagem enviada seja inferior ao LVT do processo $p_{2}$, resultando em uma anomalia de causa e efeito.

\subsection{Detecção e Tratamento de Inconsistências}

O protocolo \textit{Time Warp} não possui nenhum mecanismo que evite que uma mensagem com um \textit{timestamp} inferior ao LVT do processo receptor seja entregue. Porém, quando isto ocorre, o processo receptor identifica para qual momento no tempo de simulação (qual LVT) ele deve retornar para que haja uma consistência na simulação. Existe então um espaço no tempo de simulação que fica entre o o LVT em que o processo recebeu a mensagem \textit{straggler} e o tempo virtual referente ao \textit{timestamp} da mensagem recebida.

\emph{Definição: Janela de Retorno} - É o momento compreendido entre o LVT em que um processo lógico detecta uma anomalia e o LVT para o qual este processo realizará o \textit{rollback}.

Assim que identificado o LVT que garante a consistência da simulação para aquela processo lógico, este restaura a simulação para este este determinado tempo virtual, realizando um retorno (\textit{rollback}) da sua simulação. Para isso, todos os eventos tratados dentro da janela de retorno devem ser descartados, para serem novamente tratados pelo processo lógico que efetuou o \textit{rollback}, porém agora após o tratamento da mensagem recebida.

Conforme ilustrado na figura~\ref{fig:strag_2}, o processo $p_{1}$, cujo $LVT=9$, recebe uma mensagem com o \textit{timestamp} $t=5$, ou seja, uma mensagem para ser tratada em um tempo virtual inferior ao seu atual LVT, o que à caracteriza como mensagem \textit{straggler}. Neste caso o processo $p_{1}$ deve restornar para o $LVT=5$ para tratar a mensagem recebida. Para isto, os eventos que foram tratados dentro da janela de retorno, ou seja, os eventos $e_{7}$ e $e_{9}$ devem ser desfeitos, e novamente tratados após o tratamento da mensagem recebida.

\begin{figure}
  \centerline{\includegraphics[scale=0.4]{straggler_2.png}}
  \caption{Mensagem \textit{straggler}.}
\label{fig:strag_2}
\end{figure}

Um \textit{rollback} causado pelo recebimento de uma mensagem \textit{straggler} é chamado de \textit{rollback} primário.

Caso o processo lógico tenha enviado alguma mensagem de dentro da janela de retorno, estas mensagens devem ser canceladas pelo processo durante o \textit{rollback}. Isto é necessário porque uma mensagem enviada de dentro da janela de retorno faz parte da anomalia de causa e efeito do sistema.

Uma vez que cada mensagem enviada contém dados não confiáveis (devido à anomalia de causa e efeito), todo processo oriundo destas mensagens também deve ser desfeito. Para que isto seja feito, para cada mensagem enviada pelo processo lógico dentro da janela de retorno, uma mensagem de cancelamento deve ser enviada, indicando a invalidação da mensagem original. Esta mensagem de cancelamento é denominada anti-mensagem.

A figura~\ref{fig:strag_3} ilustra uma situação semelhante ao ilustrado pela figura~\ref{fig:strag_2}. Neste caso, além de desfazer os eventos $e_{7}$ e $e_{9}$, o processo em questão deve cancelar a mensagem enviada no tempo $t=7$ (cujo \textit{timestamp} é $t=12$) para o processo $p_{1}$. Isto é feito enviando uma anti-mensagem ao processo $p_{1}$, indicando que este deve descartar a mensagem recebida (e todo processamento relacionado à ela) por se tratar de uma mensagem que fere a consistência do sistema.

\begin{figure}
  \centerline{\includegraphics[scale=0.4]{straggler_3.png}}
  \caption{Envio de anti-mensagem para cancelamento de mensagem.}
\label{fig:strag_3}
\end{figure}

Quando um processo lógico recebe uma anti-mensagem ele verifica se o evento relacionado à esta já foi tratado ou está na lista de eventos futuros. Caso o evento esteja na lista de eventos futuros, este simplesmente é descartado (removido da lista).

Caso o evento já tenha sido tratado, existe a necessidade de se desfazer a simulação, e restaurar o processo para um tempo inferior ao \textit{timestamp} deste evento, e então descartar o evento em questão. Neste caso o processo que recebeu a anti-mensagem deverá realizar um \textit{rollback} para um LVT que garanta a consistência da simulação (ou seja, para um momento anterior ao processamento da mensagem recebida), desfazendo o tratamento do evento citado. Este \textit{rollback} realizado devido ao recebimento de uma anti-mensagem é denominado \textit{rollback} secundário.

Novamente existe a possibilidade de ter havido envio de mensagem nesta nova janela de retorno, e o processo de envio de anti-mensagens referente a cada mensagem enviada deve ser feito, a fim de neutralizar todas as mensagens enviadas decorrentes de um processamento não-consistente.

Todo o processo de envio de anti-mensagens ocorre até que o sistema encontre-se em um ponto onde não haja mais inconsistência, e a simulação é retomada.

\subsection{Considerações Finais}

O protocolo \textit{Time Warp} é uma solução que sofreu avanços ao longo do tempo desde que foi proposto em 1985. 

As utilizações de \textit{rollbacks} com intuito de recuperar sistemas inconsistentes permitem aos protocolos otimistas grande liberdade na execução de eventos, proporcionando um alto ganho em velocidade na obtenção de resultados em simulações. 

Uma grande desvantagem de um protocolo como o \textit{Time Warp} é a existência de \textit{rollbacks} secundários, que por sua vez podem disparar mais \textit{rollbacks}, transformando a recuperação do sistema em algo não muito eficiente. Assim, o surgimento de um protocolo que não utiliza anti-mensagens pode ser bastante viável. O surgimento do protocolo Rollback Solidário, proposto por~\cite{EDMARMO}, apresenta uma nova solução utilizando o conceito de checkpoints consistentes para recuperação do sistema, possibilitando que todos os processos lógicos executem um retorno ao mesmo tempo.

\section{O protocolo \textit{Rollback} Solidário}

Parágrafo 1: Apresentar a principal diferença do rollback solidário

Parágrafo 2: explicar que, em caso de erro de causa e efeito, todos os processos executam rollback em conjunto

Parágrafo 3: introduzir o conceito de checkpoint, checkpoint global e checkpoint local.

\subsection{Comportamento Geral do Protocolo Rollback solidário}

Trazer uma idéia básica do Rollback solidário em linhas gerais

\subsection{Estados Locais e Estados Globais}

Parágrafo 1: Definição de um estado local como sendo os valores das variáveis do processo em questão em um determinado tempo

Parágrafo 2: Definição de estado global como sendo um conjunto de estdaos de cada processo.

\subsection{Cortes Globais Consistentes}

Parágrafo 1: Apresentar o conceito de precedência causal

Parágrafo 2: Definir corte 

Parágrafo 3: definir corte consistente utilizando precedência causal

\subsection{Relógios Lógicos e Relógios Vetoriais}

Parágrafo 1: Apresentar o conceito de relógio lógico

Parágrafo 2: Mostrar que o relógio lógico não representa a precedência causal em um sistema distribuído

Parágrafo 3: Apresentar o conceito de relógio vetorial

\subsection{Checkpoints Globais Consistentes}

Parágrafo 1: Definir o que é um checkpoint: um ponto de retorno que foi, de alguma forma, armazenado de forma persistente

Parágrafo 2: a necessidade de se garantir que um checkpoint seja consistente

Parágrafo 3: associação de checkpoint consistente com corte consistente

\subsection{Obtenção de Checkpoint Semi-Síncrono}

Parágrafo: introduzir o processo observador

Parágrafo: Introduzir o vetor de dependência

Parágrafo: Mostrar como a troca de mensagem deve levar o vetor de dependências, e como este deve ser atualizado no processo que recebeu a mensagem

Parágrafo: mostrar como o processo observador monta uma linha de recuperação através da matriz de dependências

Parágrafo: demonstrar que um checkpoint global consistente é formado por checkpoints locais que não possuem relação causal entre si


\subsection{Tratamento dos Rollbacks na Abordagem Semi-Síncrona}

Parágrafo: Explicar o comportamento de um processo ao receber uma mensagem straggler 

Parágrafo: Falar sobre a atuação do processo observador (escolher a linha de retorno, avisar o rollback à todos em broadcast, etc.)

Parágrafo: Falar sobre o reinício da simulação após o Rollback

\subsection{Considerações Finais}

O protocolo de sincronização \textit{Rollback} Solidário apresenta uma alternativa ao \textit{Time Warp}. Sua abordagem de efetuar a restauração de todos os processos lógicos concomitantemente resulta em um ganho natural de desempenho, uma vez que é eliminada a existência de \textit{rollbacks} secundários.


\section{Balanceamento de cargas}

Parágrafo: apresentar o balanceamento de carga como fator determinante no desempenho de uma simulação distribuída

Parágrafo: Introduzir o conceito de escalonamento de processos como meio de prover o balanceamento de cargas

Parágrafo: por fim mostrar que para prover o balanceamento de cargas temos que possibilitar que um processo lógico migre de um nó do sistema para outro

\subsection{O Uso de Agentes Móveis Para Prover Mobilidade}

% Parágrafo: O que é um agente móvel

Agentes Móveis são entidades computacionais com características como autonomia, habilidades sociais, capacidade de aprendizado e, mais importante, mobilidade \cite{MOBILE-AGENTS-WIKIPEDIA}. Um agente móvel é um processo que possui a capacidade de transportar seu atual estado de um ambiente para outro, incluindo o estado de suas variáveis internas. Esta característica de mobilidade provida pelos agente móveis, de certa forma, supre uma necessidade no desenvolvimento de aplicações de simulação distribuída que pretendem prover, de alguma forma, balanceamento de cargas entre os nós do sistema.

% Parágrafo: Falar sobre a implementação feita por Antonio, Walbon e Takahashi - 2010

Aproveitando a mobilidade provida por um agente móvel, ~\cite{RIBEIROALVES} desenvolveram uma implementação do protocolo \textit{Rollback} Solidário utilizando a biblioteca \textit{Aglets}~\cite{AGLETS} de agentes móveis.

% Parágrafo: Outras implementações feitas usando agentes móveis

Outras soluções como \cite{SASSY} e \cite{}, baseiam-se nos agentes móveis para se desenvolver a aplicação de simulação distribuída, porém não necessariamente aproveitando da mobilidade proporcionada por eles para prover balanceamento de carga.

\subsection{Requisitos Para Mobilidade de um Processo Lógico}

% Parágrafo: Abstrair o conceito de ambiente e processo lógico: um ambiente abriga diversos processos lógicos
A mobilidade é a principal característica provida por um agente móvel que justifique o seu uso no desenvolvimento de uma simulação distribuída. Porém os agentes móveis possuem determinadas características que não se adequam diretamente ao sistema (conforme descrito na seção~\ref{solucoes_existentes}). Uma alternativa ao uso de agentes móveis é a criação de entidades que encapsulem os processos lógicos e supram suas necessidades básicas para o desenvolvimento da aplicação de simulação distribuída: autonomia, migração e comunicação entre os processos.

Para isto é criada uma abstração na qual os processos lógicos habitam. Essa abstração é denominada ambiente (\textit{environment}). Todo processo lógico habita um \textit{environment}, e todo todo processo lógico que habita um determinado ambiente possui a capacidade de migrar do ambiente que se encontra atualmente para um novo ambiente, e lá continuar seu processamento.

% Parágrafo: Mostrar a idéia de migrar um processo de um ambiente para outro, a fim de prover o balanceamento
Sendo assim, existe a capacidade de manipular a localização dos processos lógicos ao longo da simulação. Isso possibilita que a localização dos processos lógicos seja rearranjada conforme a necessidade da simulação ao longo do tempo, a fim de prover um melhor aproveitamento dos recursos existentes, provendo o balancemento das cargas no sistema.

O \textit{middleware} de comunicação proposto por este trabalho contempla a criação desta abstração de ambiente onde os processos lógicos residem, assim como provê uma pase para a criação destes processos lógicos.

\subsection{Considerações Finais}

O capítulo seguinte trata com detalhes o porque de se utilizar uma nova abstração para o desenvolvimento dos processos lógicos, e por que esta proposta foi preferida à utilização dos agentes móveis. O capítulo traz também a arquitetura básica de um \textit{framework} de simulação, e como o proposto \textit{middleware} de comunicação substitui o uso de agentes móveis na construção de um \textit{framework} para simulação distribuída.
