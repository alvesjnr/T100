\chapter{A linguagem \textit{Python} \label{appendix_python}}
\chapter{Diagrama de classes \label{appendix_uml_completo}}
\chapter{Distribuiçao}

O projeto \textit{framework} desenvolvido neste trabalho recebeu o nome de T100. O projeto T100, incluindo seu código fonte e toda a documentação é distribuído sobre licensa \textit{GPL - General Public License}, que dá ao seu usuário garantias de liberdade para uso, modificação e redistribuição do software como um todo, incluindo o seu código fonte.

\section{Licensa}

O projeto T100 é distribuído sobre licensa \textit{GPL v3.0}.

\subsection{\emph{License}}

\lstinputlisting[label=license,
                 caption="Licensa de uso do software"]{license.txt}

\subsection{Informações sobre a licensa}

Para mais informações sobre a lincesa \textit{GPL v3.0} visite:

http://www.gnu.org/licenses/gpl.html

\section{Disponibilidade}

O código fonte, junto aos exemplos de uso e documentação deste projeto, podem ser adquiridos através do endereço:

http://github.com/alvesjnr/T100


\chapter{Árvore de diretórios \label{appendix_tree}}

\lstinputlisting[language=Python,
                 label=source_tree,
                 caption="Árvore de diretórios do projeto."]{source_tree.txt}

\chapter{Classe com método assíncrono \label{appendix_proxy_example}}

\lstinputlisting[language=Python,
                 label=proxy_example,
                 caption="Exemplo de chamada de métodos assíncronas na classe \textit{Proxy}"]{proxy_example.py}