\chapter{Introdução}
\section{Simulação}

A simulação é uma técnica que permite prever e visualizar o comportamento de sistemas reais a partir de modelos matemáticos. As aplicações da simulação abrangem diversos benefícios, tais como: a possibilidade de antever possíveis problemas ou comportamentos indesejáveis de um sistema, auxílio na tomada de decisão sem a necessidade de intervir no sistema real, facilidade na manipulação e alteração dos modelos, economia de recursos (físicos e financeiros) durante a tomada de decisões, dentre outros.

Para utilizar a simulação é necessário construir e analisar modelos que represente o sistema. Os modelos podem ser classificados de diferentes formas. Uma classificação pode ser considerada verificando a influência ou não de variáveis aleatórias no sistema. Os sistemas são ser representados por um modelo determinístico, quando estes podem ser considerado totalmente livre de aleatoriedade, ou estocásticos, quando estes consideram aleatoriedade.

Os modelos que descrevem o comportamento através do tempo podem ser classificados como contínuos e discretos no tempo. Nos modelos de estados contínuos, as variáveis de estados variam espontaneamente. Já nos modelos de estados discretos, as mudanças ocorrem em pontos específicos e descontínuos do tempo.

\section{Objetivos}
Este trabalho tem como objetivo a criação de uma arquitetura de um \textit{middleware} para auxiliar na criação de simulações distribuídas de eventos discretos. A arquitetura proposta tem como objetivo prover ferramentas para diversas funções úteis na simulação distribuída, como troca de mensagens, além de possibilitar a migração de processos.

\section{Organização da Monografia}
Os capítulos seguintes abordam uma leve explicação do funcionamento dos protocolos de sincronização de simulação distribuída e de como o trabalho foi abordado para a sua implementação.

O capítulo dois trata da simulação de eventos discretos e de simulação distribuída. Ele inicia com a explicação das principais diferenças entre protocolos conservativos e otimistas, e em seguida traz o princípio básico de funcionamento dos protocolos \textit{Time Warp} e \textit{Rollback} Solidário.

O terceiro capítulo trata de escalonamento de processos e balanceamento de carga. Nesta parte é demonstrado que a migração de processos pode contribuir com o desempenho da simulação distribuída, o que justificaria a existência de uma ferramenta nativa na arquitetura do \textit{middleware} que proporcionasse a migração de processos.

O quarto capítulo trata a apresentação da arquitetura do \textit{middleware}, assim como o seu funcionamento.

O capítulo número cinco trata tanto da implementação de sessões críticas do middleware quanto dos testes de funcionabilidades de suas funções básicas.

Por fim, o capítulo número seis discute as conclusões obtidas e as possibilidades de continuação deste trabalho.