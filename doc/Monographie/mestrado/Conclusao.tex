\chapter{Conclusões}
A implementação do protocolo \textit{Rollback} Solidário utilizando a biblioteca de agentes móveis \textit{Aglets}, proposto nesse trabalho, trouxe algumas ferramentas para a inserção de um código de simulação distribuída sobre uma rede de computadores (sistema distribuído). A utilização de uma biblioteca de agentes móveis em Java foi útil na simplificação na comunicação e na migração dos processos. 

Durante a execução deste trabalho, várias etapas individuais foram cumpridas. Em um primeiro momento, a preocupação foi a de se compreender de forma satisfatória o funcionamento de um sistema de simulação distribuído (simulação discreta, no contexto desse trabalho). Foram introduzidos os conceitos de simulação orientada a eventos, relógios lógicos, relógios vetoriais, \textit{checkpoints}, precedência causal, mensagens e anti-mensagens, mensagens \textit{straglers}, cortes consistentes, linhas de recuperação e, finalmente, o funcionamento do protocolo \textit{Rollback} Solidário na sua forma semi-síncrona, com processo observador.

Um segundo passo foi o aprendizado da biblioteca \textit{Aglets}. Nessa etapa foram desenvolvidas pequenas aplicações utilizando-se a biblioteca \textit{Aglets} e o servidor de agentes \textit{Tahiti}. O processo de aprendizado foi baseado no manual oficial da biblioteca\cite{ManualAglets}. Foram desenvolvidas as técnicas de criação de \textit{aglets} através do servidor, criação de \textit{aglets} através de outros agentes já existentes, clonagem de agentes, migração, compreendimentos sobre o ciclo de vida de um \textit{aglet}, envio, recebimento e tratamento de mensagens, dentre outros pontos.

\section{Sobre a Implementação}
No desenvolvimento do protocolo, um primeiro obstáculo encontrado foi a impossibilidade de se invocar métodos de outros agentes. Um agente se comunica com outro exclusivamente através de mensagens, mesmo estando em um mesmo contexto. Para se solucionar este problema, foi utilizada uma mensagem de disparo de método. Junto à essa mensagem, é enviada o parâmetro a ser tratado pelo método invocado, através do argumento arg da mensagem enviada. Com isso foi possível a invocação de métodos de outros agentes, estejam eles no mesmo contexto ou até mesmo em outro contexto. 

Decorrente dessa solução de se enviar o parâmetro do método como argumento de uma mensagem, ocorreu a necessidade de se enviar por vezes objetos que nào suportavam serialização. Esses objetos não podem ser enviados como argumento de uma mensagem (é o caso do objeto \textit{AgletID}). Para isso foi utilizado o método \textit{toString()} em conjunto com o construtor \textit{AgletID(String str)}, da classe \textit{AgletID} . Com isso, o objeto \textit{AgletID} foi, no agente de origem, convertido em uma \textit{String}, enviado ao destino como \textit{String} e, ao chegar no seu agente destino, foi recriado a partir desse argumento enviado como \textit{String}.

Outros objetos, como o \textit{AgletContext}, não suportam serialização. Isso serve, de certa forma, para que em uma migração de contexto, o abjeto \textit{AgletContext} criado não venha a ser errôneamente utilizada. Uma forma contornar essa situação é a de apagar objeto \textit{AgletContext} antes de uma migração e atulizá-lo assim que o objeto chega ao seu novo contexto.

Por fim, alguns métodos utilizados no manual estão marcados como \textit{Deprecated}. Um desses métodos foi largamente utilizado nesse trabalho, o método \textit{getAgletProxy(URL,AgletID)} da classe \textit{AgletContext}. Esse método consiste em receber o ID de um agente e a URL do seu contexto e, com isso, gerar um \textit{proxy} que possibilite a comunicação entre esses agentes. O problema criado por esse método é que, quando se é requisitado um proxy de um agente que não pertence a mesma URL, durante a comunicação podem ocorrer falhas nos envios das mensagens. Para a solução desse problema foram testadas algumas alternativas:

\begin{itemize}
	\item \textbf{Utilizar um método que realize a mesma tarefa:} Não foi encontrado em toda a documentação um método que possa substituir este método em questão, sendo essa alternativa descartada;
	\item \textbf{Serializar e enviar o proxy de comunicação:} O objeto \textit{AgletProxy} não suporta serialização, o que impossibilitou essa solução;
	\item \textbf{Centralizar todo o tráfego para o processo observador:} Essa solução consiste em realizar a comunicação sempre entre o observador e um processo. Caso um processo precise enviar uma mensagem a outro, essa mensagem seria enviada ao observador que redirecionaria a mensagem ao processo em questão. Essa solução resolveria o problema de comunicação entre os processos, mas poderia representar um gargalo no sistema. Esse é um dos pontos em aberto desse trabalho.
\end{itemize}


\section{Contribuições Desse Trabalho}
Com a finalização da implementação proposta nesse trabalho, uma série de ferramentas foram concluídas para a inserção de um código de simulação distribuída sobre o protocolo \textit{Rollback} Solidário.

Os serviços de criação e despacho dos processos para seus hosts de destinos foram concluídos, assim como os processos de retirada de eventos da lista de eventos para tratamento. A inserção de um evento na lista de eventos futuros em tempo de execução, assim como a sinalização de recebimento de uma mensagem \textit{straggler} também foi concluída. Também estão finalizados os métodos para retorno à um tempo anterior da simulação, caso ocorra um \textit{rollback}.

No processo observador, todo o processo de recebimento de \textit{checkpoint}, detecção de corte consistente com base na matriz de dependência, eleição da linha de retorno e envio das mensagens de \textit{rollback} com a linha de recuperação para os processos está concluída.



\section{Trabalhos Futuros}

Com o fim da implementação desse trabalho, alguns novos pontos podem ser considerados para dar continuidade a esse projeto:
\begin{itemize}
	\item Criação de um método eficiente de envio de mensagens \textit{multicast} para os processos;
	\item Solução do problema de envio de mensagens para processos fora da mesma rede. A solução proposta de centralizar o tráfego de mensagens pode ser analizada como uma alternatica para o caso;
\end{itemize}